\appendices
%
% If you only have one appendix, you should change the above to:
%\appendix
%

\chapter{GLOSSARY AND ACRONYMS}

\begin{itemize}
\item API : ApplicationProgrammingInterface
\item ATL : Active Template Library
\item AVL : G. M. Adelson-Velskii and E. M. Landis
\item BST : Binary Search Tree
\item CAD : Computer Aided Design
\item CADD : Computer Aided Design and Drafting
\item CAM : Computer Aided Manufacturing
\item COM : Component Object Model
\item DLL : Dynamic Link Library
\item DT(P) : Delaunay Triangulation for a set P
\item DWORD : Double Word
\item ESRI : Environmental Systems Research Institute
\item GIS : Geographic Information Science
\item GPS : Global Positioning System
\item GUI : Graphical User Interface
\item IDE : Integrated Development Environment
\item MFC : Microsoft Foundation Classes
\item MSDN : Microsoft Developer Network
\item MVVM : Model-View-View Model
\item RC : Release Candidate
\item SDK : Software Development Kit
\item UI : User Interface
\item WPF : Windows Presentation Foundation
\end{itemize}

\chapter{ALGORITHM PSEUDO CODE}

\noindent
Pseudo code of the implemented algorithm: \newline

\noindent
METHOD Algorithm
    \newline \hspace*{2em}Generate random lines
    \newline \hspace*{2em}Initialize Event Queue Q using the randomly generated lines
    \newline \hspace*{2em}Initialize Status Structure $\tau$;
    \newline \newline \hspace*{2em}CALL FindIntesections with Q and $\tau$
\newline END METHOD

\noindent
\newline METHOD FindIntersections(EventQueue Q, StatusStructure $\tau$)
    \newline \hspace*{2em} Event Point P = Left most leaf in the Event Queue Q;
    \newline \hspace*{2em} WHILE( P != 0 )
        \newline \hspace*{4em} CALL HandleEventPoint with P and $\tau$
        \newline \hspace*{4em} P = Next adjacent right leaf in Event Queue Q;
    \newline \hspace*{2em} END WHILE
\newline END METHOD \newline

\noindent
\newline METHOD HandleEventPoint(EventPoint P, StatusStructure $\tau$)
    \newline \hspace*{2em} arrayU = Set of line segments whose upper end point is P;
    \newline \hspace*{2em} arrayL = Set of line segments whose lower end point is P;
    \newline \hspace*{2em} arrayC = Set of line segments which contains P somewhere in the middle; \newline
    \newline \hspace*{2em} arrayUnion = Unionof arrayU, arrayL and arrayC; \newline
    \newline \hspace*{2em} IF arrayUnion contains more than 1 element
        \newline \hspace*{4em} P is the intersection point and hence  record it;
    \newline \hspace*{2em} END IF \newline
    \newline \hspace*{2em} arrUnionLowerMiddle = Union of arrayL and arrayC;
    \newline \hspace*{2em} FOR EACH element in arrUnionLowerMiddle
        \newline \hspace*{4em} remove element from Status Structure $\tau$;
    \newline \hspace*{2em} arrUnionUpperMiddle = Union of arrayU and arrayC;
    \newline \hspace*{2em} FOR EACH element in arrUnionUpperMiddle
        \newline \hspace*{4em} remove element from Status Structure $\tau$; \newline
    \newline \hspace*{2em} If arrUnionUpperMiddle contains no elements
        \newline \hspace*{4em} LeftNode = Node to the Left of element in Status Structure $\tau$;
        \newline \hspace*{4em} RightNode = Node to the right of element in Status Structure $\tau$; \newline
        \newline \hspace*{4em} CALL FindNewEvent with LeftNode, RightNode and P;
    \newline \hspace*{2em} ELSE IF
        \newline \hspace*{4em} LeftMostSeg = Left Most segment in arrUnionUpperMiddle;
        \newline \hspace*{4em} LeftNeighbor = Left neigbour of LeftMostSeg in $\tau$; \newline
        \newline \hspace*{4em} CALL FindNewEvent with LeftNeighbor, LeftMostSeg and P \newline
        \newline \hspace*{4em} RightMostSeg = Right Most segment in arrUnionUpperMiddle;
        \newline \hspace*{4em} RightNeighbor = Right neighbor of RightMostSeg in $\tau$; \newline
        \newline \hspace*{4em} CALL FindNewEvent with RightMosstSeg, RightNeighbor and P
    \newline \hspace*{2em} END IF
\newline END METHOD \newline

\noindent
\newline METHOD FindNewEvent(leftSeg, rightSeg, P)
    \newline \hspace*{2em} IF leftSeg and rightSeg intersect below the sweep line or on it and to the right of P and intersection point in not yet present in the Event Queue Q
    \newline \hspace*{4em} Intersection Point detected and hence report it;
    \newline \hspace*{2em} END IF
\newline END METHOD \newline

\chapter{RANDOM NUMBER GENERATION CODE}

\noindent /* RAND.C: This program seeds the random-number generator with the time, then displays 10 random integers */ \newline

\noindent \#include $<$stdlib.h$>$ \newline
\noindent \#include $<$stdio.h$>$ \newline
\noindent \#include $<$time.h$>$ \newline

\noindent void main( void ) \newline
\noindent \{
   \newline \hspace*{2em} int i;
   \newline \hspace*{2em} /* Seed the random-number generator with current time so that the numbers
   \newline \hspace*{2em}    will be different every time we run */ \newline
   \newline \hspace*{2em} srand( (unsigned)time( NULL ) );
   \newline \hspace*{2em} /* Display 10 numbers. */
   \newline \hspace*{2em} for( i = 0;   i $<$ 10;i++ )
   \newline \hspace*{4em} printf( ``  \%6d$\backslash$n", rand() );
\newline \noindent \}

\noindent \newline Output \newline
\noindent \newline 60
\noindent \newline 86
\noindent \newline 22
\noindent \newline 69
\noindent \newline 79
\noindent \newline 73
\noindent \newline 10
\noindent \newline 1
\noindent \newline 83
\noindent \newline 33
