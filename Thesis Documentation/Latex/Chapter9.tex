% Note that if you want something in single space you can go back and
% forth between single space and normal space by the use of \ssp and
% \nsp.  If you want doublespacing you can use \dsp.  \nsp is normally
% 1.5 spacing unless you use the doublespace option (or savepaper
% option)
%
%(FORMAT) usually you *don't* want to mess with the spacing for your
%(FORMAT) final version.  If you think/know that the thesis template
%(FORMAT) and/or thesis style file is incorrect/incomplete, PLEASE
%(FORMAT) contact the maintainer.  THANK YOU!!!

\chapter{CONCLUSION}
\label{chap:conclusion}
% by labeling the chapter, I can refer to it later using the
% label. (\ref{chap:conclusion}, \pageref{chap:conclusion}) Latex will take care
% of the numbering.

Line Sweep Algorithm is undoubtedly the most popular and widely used algorithm in computational geometry. It can be thought of the essential baby steps required in order to complete the journey of Computational Geometry. Line Sweep algorithm technique is also used in many other computational geometry algorithms like Voronoi diagram and Delaunay triangulation. Proper understanding of the algorithm along with the internal implementation details will definitely be useful for people studying computational geometry algorithms.

I hope that this thesis research will help future students/ researchers in better understanding the algorithm and also provide a working copy of the implementation. Also, this thesis research has introduced the optimized version of the line sweep algorithm which is more efficient that the  traditional algorithm, proved in Chapter \ref{chap:resultsdiscussion}. The traditional algorithm can be replaced by the optimized version achieving the same result, but in a short time. This improvement will be particularly evident for large number of input values where the size of status structure $\tau$ is huge.

Also it can be concluded that using the right tools, methods and technologies to develop a software program can be definitely beneficial. This thesis research has explored the usage of SVN version control software, Visual Studio Integrated Development Environment (IDE) for building C++ programs, power of C++ language features such as operator overloading and templates. Also following good software engineering practices such as unit testing and following appropriate design patterns can help in achieving a better quality product which is easy to maintain. Also last but not the least this thesis research has also taught me the correct way to find, debug and fix memory leaks which is extremely useful in developing any kind of software and not just computational geometry algorithms. Thus this thesis research has taught me a lot of extra useful things besides understanding and improving the line sweep algorithm.
